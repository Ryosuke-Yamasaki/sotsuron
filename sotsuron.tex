\documentclass[a4paper, 12pt]{ltjsarticle}
\usepackage{thesis}

% 表紙のスタイル設定
\newcommand{\coverpage}{
\begin{titlepage}
\begin{center}
\vspace*{1.5cm}

{\LARGE 2024年度卒業論文}\\[2cm] % 年度タイトル

{\Huge {常微分方程式のシミュレータ教材の開発}}\\[4cm] % 論文タイトル 二行にしたい場合はバックスラッシュ2個つける

{\LARGE 指導教員 須田 宇宙 准教授}\\[2cm] % 指導教員名
{\LARGE 千葉工業大学 情報ネットワーク学科}\\[0.5cm] % 学科名

{\LARGE 須田研究室}\\[2.5cm] % 研究室名

{\LARGE {2032144} 氏名 {山﨑亮輔}} \\[1.5cm] % 学番と名前

\vfill
\end{center}

% 提出日を右下に配置
\begin{flushright}
{\LARGE 提出日 \textnormal{2025年01月10日}}\\[1.5cm] % 提出日
\end{flushright}

\vfill
\end{titlepage}
}

\begin{document}

% 表紙の挿入
\coverpage

\tableofcontents

\clearpage

\section{緒言}
ここに本文を書きます。数字の例: 12345。
\clearpage
\section{シミュレータ教材について}
理工学系の教育現場では,不可視現象を取り扱うことがある.不可視現象を取り扱う分野の学習に関して,座学のみでは高い学習効果を得ることが難しい.そのため,e-Learningコンテンツとして不可視現象をデジタル端末で可視化,可聴化する教材であるシミュレータ教材を用いた体験的な学習が有効的である.シミュレータ教材は目的に応じて次のように分類することができる.
\begin{enumerate}
\item 汎用測定機器教材\\
入力データを与えることで,一般の測定機器のように解析を行わせることを目的としている.
\item 時間変化を伸縮する教材\\
高速あるいは低速で時間変化する現象を伸縮して可視化することを目的としている.
\item 3D-CG表示教材\\
二次元の図では断片的に表すのみの現象を,3D-CGの技法を取り入れることで立体的に表現し,あらゆる角度から観察できる教材にすることを目的としている.
\item 可聴化教材\\
視覚では確認が難しい音響現象音響現象を,聴覚的に合成音で確認することを目的している.
\item 現象のアニメーション化\\
シミュレーション結果をもとに,直感的,現実的なアニメーション技法に基づきビットマップ対応で表現することを目的としている.
\end{enumerate}
\clearpage
\section{常微分方程式について}
\subsection{概要}
常微分方程式とは,変数が1つとなるような,ある関数とその導関数の関係を表す方程式のことである.また,解析的に解けるものには限りがあるため,実際には数値解法を用いることが多い.
\subsection{数値解法}
常微分方程式の数値解法は初期値問題に分類され,与えられた導関数における微小区間の面積を原関数における微小区間の変化分として原関数を求められる.代表的な解法としてオイラー法と4次ルンゲクッタ法がある.以下は1階常微分方程式の各解法である.
与えられた導関数を$\frac{dy}{dx}=f(x,y)$,初期値を$(x,y)=(x_0,y_0)$,微小区間の幅を$h$とする.
\subsection{オイラー法}
オイラー法では,現在の点を用いて次の点を求める.現在の点を$(x_n,y_n)$とすると,$(x_{n+1},y_{n+1})$は以下のようになる.
\begin{equation}
  x_{n+1}=x_n+h
\end{equation}
\begin{equation}
  y_{n+1}=y_n+hf(x_n,y_n)
\end{equation}
また,局所誤差は$O(h^2)$,大域誤差は$O(h)$である.
\clearpage
\subsection{4次ルンゲクッタ法}
4次ルンゲクッタ法では,次の補正値$k$を用いて次の点を求める.
\begin{equation}
  k_1=hf(x_n,y_n)\label{eq:3}
\end{equation}
\begin{equation}
  k_2=hf(x_n+\frac{h}{2},y_n+\frac{k_1}{2})\label{eq:4}
\end{equation}
\begin{equation}
  k_3=hf(x_n+\frac{h}{2},y_n+\frac{k_2}{2})\label{eq:5}
\end{equation}
\begin{equation}
  k_4=hf(x_n+h,y_n+k_3)\label{eq:6}
\end{equation}
\eqref{eq:3}では,現在の点を用いており,\eqref{eq:4}と\eqref{eq:5}は現在の点と次の点の中点を用いている.最後に,\eqref{eq:6}では,次の点を用いている.
現在の点を$(x_n,y_n)$とすると,$(x_{n+1},y_{n+1})$は以下のようになる.
\begin{equation}
  x_{n+1}=x_n+h
\end{equation}
\begin{equation}
  y_{n+1}=y_n+\frac{1}{6}(k_1+2k_2+2k_3+k_4)
\end{equation}
また,局所誤差は$O(h^5)$,大域誤差は$O(h^4)$である.
\clearpage
\section{開発したシミュレータ教材について}
\subsection{目的}
\subsection{使用例}
\clearpage
\section{教材の実装について}
\subsection{開発言語}
\subsubsection{HTML}
HTMLとは,Webの最も基本的な構成要素であり,Webページを作成する際に使用されるマークアップ言語である.また,正式名称はHyper Text Markup Language(ハイパーテキストマークアップランゲージ)である.HTMLの特徴としては,ハイパーテキスト機能でWebページから別のWebページに簡単に遷移することができる点である.
\subsubsection{CSS}
CSSとは,主にHTMLで記述された文書の体裁や見栄えを表現するためのスタイルシート言語である.また,正式名称はCascading Style Sheets(カスケーディングスタイルシート)である.
\subsubsection{JavaScript}
JavaScriptとは,主にWebページで使用されている,インタープリンター型の第一級関数を備えたプログラミング言語である.また,プロトタイプベースで動的な言語のため,オブジェクト指向,命令型,宣言型といったスタイルに対応している.
\clearpage
\subsection{実装機能}
\subsubsection{操作部分}
\subsubsection{テーブル表示部分}
\subsubsection{グラフ表示部分}
\subsubsection{拡大グラフ表示部分}
\subsection{評価}
\clearpage
\section{結言}
\clearpage

\bibliography{reference} % 参考文献ファイル名(.bib)
\end{document}