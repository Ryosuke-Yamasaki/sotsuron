\documentclass[a4paper, 12pt]{ltjsarticle}
\usepackage{thesis}

% 表紙のスタイル設定
\newcommand{\coverpage}{
\begin{titlepage}
\begin{center}
\vspace*{1.5cm}

{\LARGE 2024年度卒業論文}\\[2cm] % 年度タイトル

{\Huge {常微分方程式のシミュレータ教材の開発}}\\[4cm] % 論文タイトル 二行にしたい場合はバックスラッシュ2個つける

{\LARGE 指導教員 須田 宇宙 准教授}\\[2cm] % 指導教員名
{\LARGE 千葉工業大学 情報ネットワーク学科}\\[0.5cm] % 学科名

{\LARGE 須田研究室}\\[2.5cm] % 研究室名

{\LARGE {2032144} 氏名 {山﨑亮輔}} \\[1.5cm] % 学番と名前

\vfill
\end{center}

% 提出日を右下に配置
\begin{flushright}
{\LARGE 提出日 \textnormal{2025年01月10日}}\\[1.5cm] % 提出日
\end{flushright}

\vfill
\end{titlepage}
}

\begin{document}

% 表紙の挿入
\coverpage

\tableofcontents

\clearpage

\section{緒言}
ここに本文を書きます。数字の例: 12345。
\clearpage
\section{シミュレータ教材について}
\clearpage
\section{常微分方程式について}
\subsection{概要}
常微分方程式とは,変数が1つとなるような,ある関数とその導関数の関係を表す方程式のことである.また,解析的に解けるものには限りがあるため,実際には数値解法を用いることが多い.常微分方程式の数値解法は初期値問題に分類される.
\subsection{オイラー法}
\subsection{ルンゲクッタ法}
\subsection{比較}
\clearpage
\section{開発したシミュレータ教材について}
\subsection{目的}
\subsection{使用例}
\clearpage
\section{教材の実装について}
\subsection{開発言語}
\subsubsection{HTML}
HTMLとは,Webの最も基本的な構成要素であり,Webページを作成する際に使用されるマークアップ言語である.また,正式名称はHyper Text Markup Language(ハイパーテキストマークアップランゲージ)である.HTMLの特徴としては,ハイパーテキスト機能でWebページから別のWebページに簡単に遷移することができる点である.
\subsubsection{CSS}
CSSとは,主にHTMLで記述された文書の体裁や見栄えを表現するためのスタイルシート言語である.また,正式名称はCascading Style Sheets(カスケーディングスタイルシート)である.
\subsubsection{JavaScript}
JavaScriptとは,主にWebページで使用されている,インタープリンター型の第一級関数を備えたプログラミング言語である.また,プロトタイプベースで動的な言語のため,オブジェクト指向,命令型,宣言型といったスタイルに対応している.
\clearpage
\subsection{実装機能}
\subsubsection{操作部分}
\subsubsection{テーブル表示部分}
\subsubsection{グラフ表示部分}
\subsubsection{拡大グラフ表示部分}
\subsection{評価}
\clearpage
\section{結言}
\clearpage

\bibliography{reference} % 参考文献ファイル名(.bib)
\end{document}